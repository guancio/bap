\section{Functions}

In this tutorial, we show how to use the get\_functions command to
extract functions from a binary with debugging symbols.  Specifically,
we will look at the \cmdline{/bin/ls} command.  First, make sure you
are using a 32-bit executable; BAP does not support x86-64 yet.  If we
know the name of a function we want to extract, such as
\cmdline{hash\_table\_ok}, we can extract it using
\cmdline{get\_functions -unroll /bin/ls bap hash\_table\_ok}.  This
will produce a file called baphash\_table\_ok.il. We used the
\cmdline{-unroll} option to remove loops by unrolling; this is
important when generating VCs, because generating VCs is generally
only possible for acyclic programs. If we wanted to see if the
function can return zero, we could use topredicate:
\cmdline{topredicate -q -il baphash\_table\_ok.il -post 'R\_EAX:u32 ==
  0:u32' -stp-out /tmp/f -solve}.
